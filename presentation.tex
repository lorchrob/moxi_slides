\documentclass[11pt,xcolor={dvipsnames},hyperref={pdftex,pdfpagemode=UseNone,hidelinks,pdfdisplaydoctitle=true},usepdftitle=false]{beamer}
\usepackage{presentation}

\hypersetup{pdftitle={Augmenting Model-Based Instantiation with Fast Enumeration}}
\newcommand\sym[1]{\mathsf{#1}}
\newcommand\ty[1]{\mathit{#1}}

\newcommand{\pt}{\phantom{0}}
\newcommand{\ptt}{\phantom{0}\phantom{0}}
\newcommand{\pttt}{\phantom{0}\phantom{0}\phantom{0}}
\newcommand{\ptttt}{\phantom{0}\phantom{0}\phantom{0}\phantom{0}}
\newrobustcmd\B{\DeclareFontSeriesDefault[rm]{bf}{b}\bfseries} 
\usepackage{siunitx}
\usepackage{mdframed}
\usepackage{algorithmicx}
\usepackage{algpseudocode}
\usepackage[normalem]{ulem}
\usepackage{xcolor}
\newcommand{\code}[1]{{\footnotesize\texttt{#1}}}
\setbeamercovered{invisible}  
% Define a custom dark green color
\definecolor{darkgreen}{rgb}{0.0, 0.5, 0.0}
\begin{document}
% \setstretch{1.25}

\title{KMoXI: A MoXI Interface for the Kind 2 Model Checker}
%\subtitle{Extending SMT solving}

\information
%  
% []  
%
{\underline{Rob Lorch}\quad Daniel Larraz\quad Cesare Tinelli
\newline \newline   
The University of Iowa
}
%
%{TACAS 2025 \\2025-05-05, Hamilton, Canada}

\frame{\titlepage}

\begin{frame} 
    \frametitle{Background}
    \begin{itemize}
        \item Kind 2
        \begin{itemize}
            \item An SMT-based model checker for infinite-state synchronous reactive systems
            \item Traditional input language is Lustre
        \end{itemize}
        \pause
        \item MoXI
        \begin{itemize}
            \item \textbf{Mo}del e\textbf{X}change \textbf{I}nterlingua
            \item Common input/output language for model checkers; extension of SMT-LIB
        \end{itemize}
        \pause
        \item Introducing KMoXI, Kind 2 with a MoXI interface
    \end{itemize}
\end{frame}

\begin{frame} 
    \frametitle{KMoXI Architecture}
    \includegraphics[width=\textwidth]{/Users/lorchrob/Desktop/moxi_arch_1.png}
\end{frame}

\begin{frame} 
    \frametitle{KMoXI Architecture}
    \includegraphics[width=\textwidth]{/Users/lorchrob/Desktop/moxi_arch_2.png}
\end{frame}

\begin{frame} 
    \frametitle{KMoXI Architecture}
    \includegraphics[width=\textwidth]{/Users/lorchrob/Desktop/moxi_arch_3.png}
\end{frame}

\begin{frame} 
    \frametitle{KMoXI Architecture}
    \includegraphics[width=\textwidth]{/Users/lorchrob/Desktop/moxi_arch_4.png}
\end{frame}

% \begin{frame}
%     \frametitle{KMoXI Architecture}
%     \begin{itemize}
%         \item Input \texttt{.moxi} file
%         \item Lexing, parsing
%         \item Type checking 
%         \item Internal transition system with observer node
%         \item Model checking algorithms: k-induction, BMC, IC3, invariant generation
%         \item MoXI output (counterexample traces, proof certificates, and so on)
%     \end{itemize}
% \end{frame}

\begin{frame}
    \frametitle{KMoXI Features}
    \begin{itemize}
        \item We support
        \begin{itemize}
            \item \code{define-system} with all attributes, including subsystems
            \item \code{check-system}, including assumptions and (some types of) reachability queries
        \end{itemize}
    \end{itemize}
\end{frame} 

\begin{frame}
    \frametitle{KMoXI Limitations (Straightforward to Support)}
    \begin{itemize}
        \item Basic pipeline steps for MoXI-level error messages
        \begin{itemize}
            \item Syntax checks 
            \item Type checking
        \end{itemize} 
        \pause
        \item Enums \newline 
        {\footnotesize \texttt{(declare-enum-sort Color (red green blue))}}
        \pause
        \item User-defined functions
        \item Uninterpreted functions. Challenge: sharing a model across traces in \code{:queries} case
    \end{itemize}
\end{frame}

\begin{frame}
    \frametitle{KMoXI Limitations (Straightforward to Support)}
    \begin{itemize}
        \item Condensed output (in general, flags) \newline 
        {\footnotesize \texttt {
                (0 (var1 true) (var2 true) (var3 true)) \newline
                (1 (var2 false))}} \pause 
        \item Multiple queries, as long as no uninterpreted functions \newline 
        {\footnotesize \texttt {:queries ((q\_1 (r\_1)) ... (q\_n (r\_n)))}}
        \item Rely on the user to include variable renamings in \code{check-system}
        \item Current conditions: internal technical limitations
        %need multiple versions of the internal transition system
    \end{itemize}
\end{frame}

\begin{frame}
    \frametitle{KMoXI Difficult Limitations}
    \begin{itemize}

        % rch_1  
        % rch_1_seen -> true true true  true
        % rch_2  
        % rch_2_seen                    true

        \item Multiple reachability checks in the same query \newline \pause
        \code{:query (q (r\_1 ... r\_n))}, \newline has semantics (assuming no assumptions) \newline 
        $I_s \land T_s \land \mathsf{Eventually}\; r_1 \land \dots \land \mathsf{Eventually}\; r_n$ \pause
        \item Kind 2 does not support liveness properties \pause
        \begin{itemize}
        \item Fairness conditions \pause
        \end{itemize}
    \end{itemize}
\end{frame}

\begin{frame}
    \frametitle{Specification Rough Edges}
    \begin{itemize}
        \item Underspecification of subsystem variable names/scopes in counterexamples \pause
        \begin{itemize}
            \item We use double colon syntax to give scope, e.g. \code{main::subsys::var} \pause
            \item Some variables are repeated in the output \newline 
            \code {
                (define-system main ... \newline 
                    \hspace{1cm} :local ((local\_x Bool)  \newline 
                    :subsys (call\_subsys (subsys local\_x ... \newline 
                (define-system subsys \newline
                    :input ((x Bool)) \newline
            } 
            We get both \code{main::subsys::x} and \code{main::local\_x}
        \end{itemize}
    \end{itemize}
\end{frame}

\begin{frame}
    \frametitle{Specification Rough Edges}
    \begin{itemize}
        \item Reachability conditions with primed variables \pause
        \begin{itemize}
            \item Consider \code{rch := x' > x} with the empty transition system \newline 
            \code{(0 (x 0) (rch true)) \newline
            (1 (x 1) (rch ???))}
        \end{itemize} \pause
            \item Distinguishing \code{define-system} and \code{check-system} variable scoping \newline 
            \code{(define-system main :input ((rch\_1 Bool)) ...  \newline
         (check-system main :reachable (rch\_1 ...) ...}
    \end{itemize}
\end{frame}

\begin{frame}
    \frametitle{Specification Rough Edges}
    \begin{itemize}
        \item Deadlock checks \pause
        \item Ordering of options for ease in parsing\newline 
        \code{\{:input, :output, :local\}, \newline :subsys, \newline \{:init, :trans, :inv\}} \pause
        \item Printing values of variables with array types
        \begin{itemize}
            \item \code{(get-value ...)} for MoXI?
            \item Need trail name, timestep, and term as inputs
        \end{itemize}
    \end{itemize}
\end{frame}

\begin{frame}
    \frametitle{Thanks!}
    \begin{itemize}
        \item Any questions?
        \item \footnotesize{\url{https://github.com/kind2-mc/kind2}} \newline with \code{make kmoxi}
    \end{itemize}
\end{frame} 


\end{document}